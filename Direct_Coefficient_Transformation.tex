\documentclass[]{article}
\usepackage{lmodern}
\usepackage{amssymb,amsmath}
\usepackage{ifxetex,ifluatex}
\usepackage{fixltx2e} % provides \textsubscript
\ifnum 0\ifxetex 1\fi\ifluatex 1\fi=0 % if pdftex
  \usepackage[T1]{fontenc}
  \usepackage[utf8]{inputenc}
\else % if luatex or xelatex
  \ifxetex
    \usepackage{mathspec}
  \else
    \usepackage{fontspec}
  \fi
  \defaultfontfeatures{Ligatures=TeX,Scale=MatchLowercase}
\fi
% use upquote if available, for straight quotes in verbatim environments
\IfFileExists{upquote.sty}{\usepackage{upquote}}{}
% use microtype if available
\IfFileExists{microtype.sty}{%
\usepackage{microtype}
\UseMicrotypeSet[protrusion]{basicmath} % disable protrusion for tt fonts
}{}
\usepackage[margin=1in]{geometry}
\usepackage{hyperref}
\hypersetup{unicode=true,
            pdftitle={Direct Coefficient Recentering and Rescaling},
            pdfauthor={Doug Hemken},
            pdfborder={0 0 0},
            breaklinks=true}
\urlstyle{same}  % don't use monospace font for urls
\usepackage{graphicx,grffile}
\makeatletter
\def\maxwidth{\ifdim\Gin@nat@width>\linewidth\linewidth\else\Gin@nat@width\fi}
\def\maxheight{\ifdim\Gin@nat@height>\textheight\textheight\else\Gin@nat@height\fi}
\makeatother
% Scale images if necessary, so that they will not overflow the page
% margins by default, and it is still possible to overwrite the defaults
% using explicit options in \includegraphics[width, height, ...]{}
\setkeys{Gin}{width=\maxwidth,height=\maxheight,keepaspectratio}
\IfFileExists{parskip.sty}{%
\usepackage{parskip}
}{% else
\setlength{\parindent}{0pt}
\setlength{\parskip}{6pt plus 2pt minus 1pt}
}
\setlength{\emergencystretch}{3em}  % prevent overfull lines
\providecommand{\tightlist}{%
  \setlength{\itemsep}{0pt}\setlength{\parskip}{0pt}}
\setcounter{secnumdepth}{0}
% Redefines (sub)paragraphs to behave more like sections
\ifx\paragraph\undefined\else
\let\oldparagraph\paragraph
\renewcommand{\paragraph}[1]{\oldparagraph{#1}\mbox{}}
\fi
\ifx\subparagraph\undefined\else
\let\oldsubparagraph\subparagraph
\renewcommand{\subparagraph}[1]{\oldsubparagraph{#1}\mbox{}}
\fi

%%% Use protect on footnotes to avoid problems with footnotes in titles
\let\rmarkdownfootnote\footnote%
\def\footnote{\protect\rmarkdownfootnote}

%%% Change title format to be more compact
\usepackage{titling}

% Create subtitle command for use in maketitle
\newcommand{\subtitle}[1]{
  \posttitle{
    \begin{center}\large#1\end{center}
    }
}

\setlength{\droptitle}{-2em}

  \title{Direct Coefficient Recentering and Rescaling}
    \pretitle{\vspace{\droptitle}\centering\huge}
  \posttitle{\par}
    \author{Doug Hemken}
    \preauthor{\centering\large\emph}
  \postauthor{\par}
      \predate{\centering\large\emph}
  \postdate{\par}
    \date{2018-12-04}


\begin{document}
\maketitle

{
\setcounter{tocdepth}{2}
\tableofcontents
}
\hypertarget{introduction}{%
\section{Introduction}\label{introduction}}

For analysts working with linear models, recentering and rescaling the
variables under analysis is such a routine task it hardly garners
attention. In fields where there are no natural, physical units of
measurement - education and psychology, to name two - it is common
practice to refer to standardized units of measure. It is not uncommon
to see analysts fit and report the same model in both the original and
standardized units of measurement.

\hypertarget{the-problem-higher-order-models}{%
\section{The Problem: Higher Order
Models}\label{the-problem-higher-order-models}}

For additive models - models with intercepts and slopes of single
variables to polynomial degree one - the analyst can directly transform
the coefficients in the model via a classic formula that appears in most
textbooks. Consider, for example, the regression model
\[y = \beta_0 + \beta_1x_1 + \beta_2x_2\] where \(x_1\), \(x_2\), and
\(y\) are all continuous variables. If we transform the data so that all
the variables are centered, the transformed coefficients for this model
are given by \(\beta_0=0\) and \(\beta_i=\beta_i\) for \(i\ge 1\). If we
further transform the data so that all variables are standardized, the
transformed coefficients for this model are given by \(\beta_0=0\) and
\[\beta_i=\frac{\sigma_{x_i}}{\sigma_y}\beta_i\]

However, once interaction terms and higher order polynomial terms appear
in a model, the classic formula requires centering and rescaling higher
order terms using means and standard deviations of the higher order data
vectors, independent of the rescaling of lower order terms. This
produces coefficients that are difficult to interpret because they are
on different scales. While this can be useful for some purposes, such as
calculating predicted values and residuals, standard practice where the
coefficients are to be interpreted is to recalculate the data, then
refit the model.

Available software calculates standardized coefficients directly using
the classic approach, perhaps as a legacy of the sweep operations of the
1970s. Refitting the model to recalculated data has the advantage that
software also produces a variance-covariance matrix appropriate to the
transformed coefficients, and sets up the software for post-estimation
operations with the transformed model.

It seems to be little appreciated that the coefficients for recentered
and rescaled (including standardized) models can be easily calculated
directly.

\hypertarget{direct-transformation}{%
\section{Direct Transformation}\label{direct-transformation}}

\hypertarget{one-variable}{%
\subsection{One Variable}\label{one-variable}}

Transforming the coefficients and the variance-covariance matrix of a
linear model with a single continous outcome and a single continuous
predictor is straighforward. Consider the simple model
\[y = b_0 + b_1x\] or in the usual matrix form \[Y=X\beta\]

If we wish to recenter our model in terms of \(x_\delta=x-\mu_x\), an
arbitrarily recentered \(x\), we can do so without calculating the
\(x_\delta\). We use a linear transformation \(C\) to map the vector of
coefficients \(\beta\) to a vector of centered coefficients,
\(\beta_\delta\). \[\beta^\delta=C\beta\] where \(C\) takes the form
\[C=\begin{bmatrix}1 & \mu \\ 0 & 1 \end{bmatrix}\] Rescaling in terms
of \(x_z=\frac{x_\delta}{\sigma}\) can be done directly with the linear
transformation \[S=\begin{bmatrix}1 & 0 \\ 0 & \sigma \end{bmatrix}\] To
standardize the coefficients, we recenter, then rescale. In one step
this is
\[Z = S \times C =\begin{bmatrix}1 & \mu \\ 0 & \sigma \end{bmatrix}\]
To fully standardize this model, the final step is to standardize \(y\),
\(y_z=(y-\mu_y)/\sigma_y\). This requires adjusting \(\beta_0^z\) by
\(\mu_y\), which in this simple case leaves \(\beta_0^z=0\), and
dividing \(\beta^z/\sigma_y\). More complicated models also require
these two final operations for full standardization. It will simplify
further discussion to drop consideration of this.

\hypertarget{example-one-variable-recentering}{%
\subsubsection{Example: One Variable
Recentering}\label{example-one-variable-recentering}}

It is worth noting that ``recentering'' and ``rescaling'' may be done
with any arbitrary constants, although it is perhaps most often done
with a sample mean and sample standard deviation. However, this same
approach would hold for converting a model where \(x\) is expressed, for
example, in degrees Fahrenheit to one expressed in degrees Centrigrade.

\begin{verbatim}
example <- lm(mpg ~ wt, data=mtcars)
\end{verbatim}

Here the coefficients are recentered \emph{as if} the \(x\) variable
\texttt{wt} were recentered to the sample mean.

\begin{verbatim}
C <- matrix(c(1,0,mean(mtcars$wt),1), ncol=2)
C%*%coef(example)
\end{verbatim}

\begin{verbatim}
      [,1]
[1,] 20.09
[2,] -5.34
\end{verbatim}

We can check that this agrees with recentering the data, then refitting
the model.

\begin{verbatim}
wtcentered <- mtcars$wt - mean(mtcars$wt)
check <- lm(mpg ~ wtcentered, data=mtcars)
coef(check)
\end{verbatim}

\begin{verbatim}
(Intercept)  wtcentered 
      20.09       -5.34 
\end{verbatim}

We can also use the same recentering matrix to transform the
variance-covariance matrix of the coefficients.

\begin{verbatim}
C%*%vcov(example)%*%t(C)
\end{verbatim}

\begin{verbatim}
     [,1]  [,2]
[1,] 0.29 0.000
[2,] 0.00 0.313
\end{verbatim}

\begin{verbatim}
vcov(check)
\end{verbatim}

\begin{verbatim}
            (Intercept) wtcentered
(Intercept)        0.29      0.000
wtcentered         0.00      0.313
\end{verbatim}

\begin{verbatim}
# check equality
norm(C%*%vcov(example)%*%t(C)-vcov(check), "F")
\end{verbatim}

\begin{verbatim}
[1] 4.48e-16
\end{verbatim}

A change of basis for the column space of \(X\) induces a change of
basis for the column space of the coefficient vector, and a change of
basis for the row and column space of the variance-covariance matrix.
\(C\), \(S\), and \(Z\) are change of basis transformations.

\hypertarget{two-variables}{%
\subsection{Two Variables}\label{two-variables}}

Now consider a model with two continuous independent variables and an
interaction term, so the columns of \(X\) are
\(\begin{bmatrix} 1_n &x_1 &x_2 &x_1x_2 \end{bmatrix}\). We compose the
coefficient change of basis from the two simple transformations as a
direct product {[}reference: Haberman?{]}. Denote
\[C_1=\begin{bmatrix}1 & \mu_1 \\ 0 & 1 \end{bmatrix}\]
\[C_2=\begin{bmatrix}1 & \mu_2 \\ 0 & 1 \end{bmatrix}\] Then
\[C = C_2 \otimes C_1 = \begin{bmatrix} 1 & \mu_1 &\mu_2 &\mu_2\mu_1 \\
  0 &1 &0 &\mu_2 \\ 0 &0 &1 &\mu_1 \\ 0 &0 &0 &1 \end{bmatrix}\] The
classic standardization formula skips recentering in adjusting the
coefficients.

While this is simple to produce in theory, in practice attention must be
given to the column ordering: to use \(C\) we must recognize that the
column space is ordered
\(\begin{bmatrix} 1_n &x_1 &x_2 &x_1x_2 \end{bmatrix}\), so the order
must match that of the coefficient vector and the variance-covariance
matrix, perhaps through permutation.

While in general \(C_1 \otimes C_2 \neq C_2 \otimes C_1\), for any such
operation there exists a permutation, \(P\), such that
\(C_1 \otimes C_2 = P^T(C_2 \otimes C_1)P\). As long as we include
simple recentering and rescaling matrices for every variable used in our
coefficient terms, up to a final permutation their order does not
matter.

\hypertarget{example-two-variable-recentering}{%
\subsubsection{Example: Two Variable
Recentering}\label{example-two-variable-recentering}}

In order to build a recentering matrix, then, we need to collect a
labelled vector of recentering constants, and an ordered vector of
coefficient terms.

\begin{verbatim}
source("stdParm functions.r")

ex2 <- lm(mpg ~ wt*disp, data=mtcars)        # the base model
x.means <- colMeans(mtcars[,c("wt","disp")]) # recentering constants (means)
b.terms <- names(coef(ex2))                  # coefficients/terms

C <- recentering.matrix(x.means, b.terms)
C
\end{verbatim}

\begin{verbatim}
            (Intercept)   wt disp wt:disp
(Intercept)           1 3.22  231  742.29
wt                    0 1.00    0  230.72
disp                  0 0.00    1    3.22
wt:disp               0 0.00    0    1.00
\end{verbatim}

This, then, is what we use to produce recentered coefficients, and the
accompanying variance-covariance matrix.

\begin{verbatim}
C %*% coef(ex2)
\end{verbatim}

\begin{verbatim}
               [,1]
(Intercept) 18.8695
wt          -3.7950
disp        -0.0187
wt:disp      0.0117
\end{verbatim}

\begin{verbatim}
C %*% vcov(ex2) %*% C
\end{verbatim}

\begin{verbatim}
            (Intercept)      wt    disp  wt:disp
(Intercept)    -0.67299 -1.8682 -155.27  -431.03
wt             -1.84937 -4.8817 -426.70 -1126.34
disp            0.00716  0.0165    1.65     3.82
wt:disp         0.00826  0.0237    1.90     5.47
\end{verbatim}

\hypertarget{example-two-variable-rescaling}{%
\subsection{Example: Two Variable
Rescaling}\label{example-two-variable-rescaling}}

Rescaling is just as easy. Here,
\[S_1=\begin{bmatrix}1 &0 \\ 0 &\sigma_1 \end{bmatrix}\]
\[S_2=\begin{bmatrix}1 &0 \\ 0 &\sigma_2 \end{bmatrix}\] Then
\[S = S_2 \otimes S_1 = \begin{bmatrix} 1 &0 &0 &0 \\
  0 &\sigma_1 &0 &0 \\ 0 &0 &\sigma_2 &0 \\ 0 &0 &0 &\sigma_2\sigma_1 \end{bmatrix}\]
While the classic standardization formula takes a similar diagonal form,
the rescaling constants for higher order terms (\(>1\)) are different.

Rescaling can be useful independent of recentering, for example, to
rescale from United States customary units to SI units where the zero of
each scale remains unchanged.

\begin{verbatim}
# pounds to kilograms, and cubic inches to liters
x.scales <- c(1/453.592, 61.024)
names(x.scales) <- c("wt", "disp")
  
S <- recentering.matrix(x.scales, b.terms, type="scale")
S
\end{verbatim}

\begin{verbatim}
            (Intercept)     wt disp wt:disp
(Intercept)           1 0.0000    0   0.000
wt                    0 0.0022    0   0.000
disp                  0 0.0000   61   0.000
wt:disp               0 0.0000    0   0.135
\end{verbatim}

Check the coefficients:

\begin{verbatim}
wtkg  <- mtcars$wt*453.592  # 1000 lbs to kg
displ <- mtcars$disp/61.024 # cu.in. to liters

ex3 <- lm(mpg~wtkg*displ, data=mtcars)

ex3coefs <- cbind(S %*% coef(ex2),coef(ex3))
colnames(ex3coefs) <- c("Direct","Data Trans.")
ex3coefs
\end{verbatim}

\begin{verbatim}
              Direct Data Trans.
(Intercept) 44.08200    44.08200
wt          -0.01432    -0.01432
disp        -3.43920    -3.43920
wt:disp      0.00157     0.00157
\end{verbatim}

Compare the variance-covariance matrices:

\begin{verbatim}
norm(S %*% vcov(ex2) %*% S-vcov(ex3), "F")
\end{verbatim}

\begin{verbatim}
[1] 7.26e-15
\end{verbatim}

\hypertarget{interaction-terms-factorial-models-and-direct-products}{%
\section{Interaction Terms, Factorial Models, and Direct
Products}\label{interaction-terms-factorial-models-and-direct-products}}

To understand the use of the direct product here, we need to consider
the relationship between the \emph{data} space and the \emph{parameter}
space of a model. In all of the models considered here, the column space
of the parameters is a subset of the outer product of the columns space
of the data, including a column for the intercept term.

The data are modeled in an outer product of the data column space. In
transforming the coefficients, we need the same vector space for both
columns and rows - the Kronecker operator provides this twofold outer
product very neatly.

We may consider a model that includes only the mean of the response as a
zero-order model, sometimes called an intercept-only model. A model with
means of several categories, parameterized as a mean and offsets to that
mean, is a model with multiple intercepts (only), and is also a
zero-order model. Classical ANOVA models are zero-order models.

A model of a response with a continuous variable includes both an
intercept and a slope. This is a first-order model, with a zero-order
term and a first-order term. Adding categorical variables to the model
adds more intercepts, or zero-order terms. Adding continuous variables
adds more slopes, or first-order terms. Such additive models are all
first-order models.

An interaction term is formed as the product of two variables. A product
of categorical variables adds intercepts to the model. The interaction
of a categorical variable and a continuous variable adds slopes to the
model. In either case, the order of the model remains the same. But the
interaction formed from the product of two continuous variables adds a
second-order term to a model, a curvature.

A factorial model is formed by adding all the products of all the zero-
and first-order terms, in all combinations. If we think of the terms in
a model as its column space, then any linear model resides in a subspace
of the factorial column space. The columns of any linear model are a
subset of the columns in a full-factorial model.

Finally, transformations of the parameters are measured within the
parameter space. But transformations that reflect changes in the data
space are an outer product. Transformations from one basis for
parameters to another uses this outer product for both its column space
and row space. Kronecker operations give us just this result.

\hypertarget{less-than-full-factorial-models}{%
\section{Less-Than Full Factorial
Models}\label{less-than-full-factorial-models}}

An outer product with terms zeroed out.

In practice, many models with interaction or polynomial terms are not
full factorial models. To derive the correct recentering and rescaling
matrices means realizing that some of the elements of \(\beta\) are
\(0\).

\hypertarget{dropping-higher-order-terms}{%
\subsection{Dropping Higher Order
Terms}\label{dropping-higher-order-terms}}

Consider again the additive model
\[y = \beta_0 + \beta_1x_1 + \beta_2x_2\]

The recentering matrix for \(x_1\) and \(x_2\) is as given above,
however we have
\[\begin{bmatrix}\beta_0^\delta \\ \beta_1^\delta \\ \beta_2^\delta \\ \beta_3^\delta \end{bmatrix}=
\begin{bmatrix} 1 & \mu_1 &\mu_2 &\mu_2\mu_1 \\
  0 &1 &0 &\mu_2 \\ 0 &0 &1 &\mu_1 \\ 0 &0 &0 &1 \end{bmatrix}
\begin{bmatrix}\beta_0 \\ \beta_1 \\ \beta_2 \\ 0 \end{bmatrix}\] This
simplifies to
\[\begin{bmatrix}\beta_0^\delta \\ \beta_1^\delta \\ \beta_2^\delta  \\ \beta_3^\delta \end{bmatrix}=
\begin{bmatrix} 1 & \mu_1 &\mu_2 \\
  0 &1 &0 \\ 0 &0 &1 \\ 0 &0 &0 \end{bmatrix}
\begin{bmatrix}\beta_0 \\ \beta_1 \\ \beta_2 \end{bmatrix}\] Not
surprisingly, this leaves \(\beta_3^\delta=0\), and we can further
simplify
\[\begin{bmatrix}\beta_0^\delta \\ \beta_1^\delta \\ \beta_2^\delta \end{bmatrix}=
\begin{bmatrix} 1 & \mu_1 &\mu_2 \\
  0 &1 &0 \\ 0 &0 &1 \end{bmatrix}
\begin{bmatrix}\beta_0 \\ \beta_1 \\ \beta_2 \end{bmatrix}\] In other
words, we end up with the only tranformation being to \(\beta_0\), which
had to be the result for a recentered additive model. Following this
approach, we can derive our classic standardization formula for additive
models as well.

\hypertarget{dropping-lower-order-terms}{%
\subsection{Dropping Lower Order
Terms}\label{dropping-lower-order-terms}}

Another point worth considering is the effect of recentering variables
on a model where a lower-order term has been dropped beneath a
higher-order term. Consider the model
\[y = \beta_0 + \beta_1x_1 + \beta_3x_1x_2\] where the term
\(\beta_2x_2\) has been dropped, setting \(\beta_2=0\).

Here, our coefficient transformation looks like
\[\begin{bmatrix}\beta_0^\delta \\ \beta_1^\delta \\ \beta_2^\delta \\ \beta_3^\delta \end{bmatrix}=
\begin{bmatrix} 1 & \mu_1 &\mu_2 &\mu_2\mu_1 \\
  0 &1 &0 &\mu_2 \\ 0 &0 &1 &\mu_1 \\ 0 &0 &0 &1 \end{bmatrix}
\begin{bmatrix}\beta_0 \\ \beta_1 \\ 0 \\ \beta_3 \end{bmatrix}\] We can
simplify this somewhat as
\[\begin{bmatrix}\beta_0^\delta \\ \beta_1^\delta \\ \beta_2^\delta \\ \beta_3^\delta \end{bmatrix}=
\begin{bmatrix} 1 & \mu_1 &\mu_2\mu_1 \\
  0 &1 &\mu_2 \\ 0 &0 &\mu_1 \\ 0 &0 &1 \end{bmatrix}
\begin{bmatrix}\beta_0 \\ \beta_1 \\ \beta_3 \end{bmatrix}\] But here we
see that our recentered model gains a term and a coefficient,
\(\beta_2^\delta (=\mu_1\beta_3)\)!

The highest order term in which a variable appears is always unchanged
by recentering; lower order terms change when any of the \emph{other}
variables in a higher order term which includes the variables in the
lower order term are recentered. Going back to the additive model,
consisting of only first-order (slope) and zero-order (intercept) terms,
we see that only the intercept changes when the first order \(x_i\) are
recentered.

If we build recentering and rescaling matrices variable by variable, we
can use less-than-full factorial combinations as building blocks. That
is to say, we could build a matrix for a full-factorial model and then
drop columns for unused terms, or we could approach this piecemeal.

{[}Checking for missing lower order terms in not currently implemented.
However, this should be easy to accomplish: drop columns in a full
factorial \(C\) or \(S\) not included in the coefficient vector
(i.e.~not among the terms), then check for rows that are zero vectors
and drop (only) those.{]}

\hypertarget{untransformed-variables}{%
\subsection{Untransformed Variables}\label{untransformed-variables}}

It may be that the analyst wishes to leave some variables untransformed.
One way to view this is that the recentering constant \(\mu=0\) and the
rescaling constant \(\sigma=1\). So the ``transformation'' for this
variable is just the identity matrix
\[S=C=\begin{bmatrix}1 & 0 \\ 0 &1 \end{bmatrix}\] This leads to a
simplification of the full factorial transformation matrix in terms of
direct sums. If we have \(C_2=I_2\), then
\[C_2 \otimes C_1 = \begin{bmatrix}C_1 &0 \\ 0 &C_1\end{bmatrix} = C_1 \oplus C_1 \]

\hypertarget{polynomial-terms}{%
\section{Polynomial Terms}\label{polynomial-terms}}

An outer product with terms collected.

Like less-than-full factorial models, models with polynomial terms are
worth a little extra scrutiny. Consider the model
\[y = \beta_0 + \beta_1x + \beta_3x^2\] If we rewrite this as
\[y = \beta_0 + \beta_1x + \beta_2x +\beta_3xx\] it looks like a
factorial model. But all the effect of \(x\) is collected in a single
term, \(\beta_1\), so \(\beta_2=0\). This is perhaps easier to see if we
look at the recentering transformation
\[\begin{bmatrix}\beta_0^\delta \\ \beta_1^\delta \\ \beta_2^\delta \\ \beta_3^\delta \end{bmatrix}=
\begin{bmatrix} 1 & \mu &\mu &\mu\mu \\
  0 &1 &0 &\mu \\ 0 &0 &1 &\mu \\ 0 &0 &0 &1 \end{bmatrix}
\begin{bmatrix}\beta_0 \\ \beta_1 \\ \beta_2 \\ \beta_3  \end{bmatrix}\]
If we take \(\beta_2=0\), then we can simplify as before
\[\begin{bmatrix}\beta_0^\delta \\ \beta_1^\delta \\ \beta_2^\delta \\ \beta_3^\delta \end{bmatrix}=
\begin{bmatrix} 1 & \mu &\mu\mu \\
  0 &1 &\mu \\ 0 &0 &\mu \\ 0 &0 &1 \end{bmatrix}
\begin{bmatrix}\beta_0 \\ \beta_1 \\ \beta_3  \end{bmatrix}\] But now
\(\beta_1^\delta\) and \(\beta_2^\delta\) both have part of the effect
of a recentered \(x_1\). If we collect these terms we end up with
\[\begin{bmatrix}\beta_0^\delta \\ \beta_1^\delta \\ \beta_3^\delta \end{bmatrix}=
\begin{bmatrix} 1 & \mu &\mu\mu \\
  0 &1 &2\mu \\ 0 &0 &1 \end{bmatrix}
\begin{bmatrix}\beta_0 \\ \beta_1 \\ \beta_3  \end{bmatrix}\] Models
that include terms to any polynomial degree can be handled in this
manner.

Polynomial models where higher degree terms are included while dropping
lower degree terms, when recentered, will have the lower order terms
re-emerge, just as we saw with less-than-full factorial models with
dropped lower order terms.

\hypertarget{categorical-terms}{%
\section{Categorical terms}\label{categorical-terms}}

In practice there are a number of approaches used for categorical
variables, i.e. collections of indicator/contrast variables.

\begin{itemize}
\tightlist
\item
  Leaving the intercept terms in reference coding amounts to leaving the
  coefficients for indicators untransformed, as described previously.
\item
  Standardizing each term as a z-score is equivalent to treating each
  category in the same manner as continuous variables, as described
  previously.
\item
  Transforming to coding other than reference coding is again a
  ``recentering'' change of basis in that it changes where we find zero
  in the parameter space. Here, however, we need another simple
  transformation.
\end{itemize}

For example, the general form of the grand-mean recentering matrix for a
categorical variable with \(k\) categories is \[
\begin{bmatrix}
1 &1/k     &\cdots &1/k \\
0 &(k-1)/k &\cdots &-1/k \\
0 &-1/k    &\cdots &-1/k \\
\vdots &\vdots &\vdots &\vdots \\
0 &-1/k    &\cdots &(k-1)/k
\end{bmatrix}
\]

This is then used in the same way as previously discussed recentering
transformations. (The first category remains the dropped column in this
transformation.)

\begin{verbatim}
C1 <- ref.to.gm(3)
rownames(C1) <- colnames(C1) <- c("(Intercept)", "cyl6", "cyl8")
C1
\end{verbatim}

\begin{verbatim}
            (Intercept)   cyl6   cyl8
(Intercept)           1  0.333  0.333
cyl6                  0  0.667 -0.333
cyl8                  0 -0.333  0.667
\end{verbatim}

Combined with our transformation matrix for a continuous variable

\begin{verbatim}
wtmean <- mean(mtcars$wt)
names(wtmean) <- "wt"
C2 <- mean.to.matrix(wtmean)
C <- kron(C2,C1)
C
\end{verbatim}

\begin{verbatim}
            (Intercept)   cyl6   cyl8   wt cyl6:wt cyl8:wt
(Intercept)           1  0.333  0.333 3.22   1.072   1.072
cyl6                  0  0.667 -0.333 0.00   2.145  -1.072
cyl8                  0 -0.333  0.667 0.00  -1.072   2.145
wt                    0  0.000  0.000 1.00   0.333   0.333
cyl6:wt               0  0.000  0.000 0.00   0.667  -0.333
cyl8:wt               0  0.000  0.000 0.00  -0.333   0.667
\end{verbatim}

This converts an uncentered model with reference coding for the
indicators to centered \texttt{wt} and grand-mean centered \texttt{cyl}.

\begin{verbatim}
cylf <- as.factor(mtcars$cyl)
excat <- lm(mpg ~ cylf*wt, data=mtcars)
summary(excat)
\end{verbatim}

\begin{verbatim}

Call:
lm(formula = mpg ~ cylf * wt, data = mtcars)

Residuals:
   Min     1Q Median     3Q    Max 
-4.151 -1.380 -0.639  1.494  5.252 

Coefficients:
            Estimate Std. Error t value Pr(>|t|)    
(Intercept)    39.57       3.19   12.39  2.1e-12 ***
cylf6         -11.16       9.36   -1.19  0.24358    
cylf8         -15.70       4.84   -3.24  0.00322 ** 
wt             -5.65       1.36   -4.15  0.00031 ***
cylf6:wt        2.87       3.12    0.92  0.36620    
cylf8:wt        3.45       1.63    2.12  0.04344 *  
---
Signif. codes:  0 '***' 0.001 '**' 0.01 '*' 0.05 '.' 0.1 ' ' 1

Residual standard error: 2.45 on 26 degrees of freedom
Multiple R-squared:  0.862, Adjusted R-squared:  0.835 
F-statistic: 32.4 on 5 and 26 DF,  p-value: 2.26e-10
\end{verbatim}

\begin{verbatim}
C %*% coef(excat)
\end{verbatim}

\begin{verbatim}
              [,1]
(Intercept) 19.227
cyl6         0.237
cyl8        -2.413
wt          -3.540
cyl6:wt      0.760
cyl8:wt      1.347
\end{verbatim}

\begin{verbatim}
contrasts(cylf) <- contr.sum
coef(lm(mpg~cylf*wtcentered, data=mtcars)) # note different dropped level
\end{verbatim}

\begin{verbatim}
     (Intercept)            cylf1            cylf2       wtcentered 
          19.227            2.176            0.237           -3.540 
cylf1:wtcentered cylf2:wtcentered 
          -2.107            0.760 
\end{verbatim}


\end{document}

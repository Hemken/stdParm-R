\documentclass[]{article}
\usepackage{lmodern}
\usepackage{amssymb,amsmath}
\usepackage{ifxetex,ifluatex}
\usepackage{fixltx2e} % provides \textsubscript
\ifnum 0\ifxetex 1\fi\ifluatex 1\fi=0 % if pdftex
  \usepackage[T1]{fontenc}
  \usepackage[utf8]{inputenc}
\else % if luatex or xelatex
  \ifxetex
    \usepackage{mathspec}
  \else
    \usepackage{fontspec}
  \fi
  \defaultfontfeatures{Ligatures=TeX,Scale=MatchLowercase}
\fi
% use upquote if available, for straight quotes in verbatim environments
\IfFileExists{upquote.sty}{\usepackage{upquote}}{}
% use microtype if available
\IfFileExists{microtype.sty}{%
\usepackage{microtype}
\UseMicrotypeSet[protrusion]{basicmath} % disable protrusion for tt fonts
}{}
\usepackage[margin=1in]{geometry}
\usepackage{hyperref}
\hypersetup{unicode=true,
            pdftitle={Factorial Models, Kronecker Products, and Interaction Terms},
            pdfauthor={Doug Hemken},
            pdfborder={0 0 0},
            breaklinks=true}
\urlstyle{same}  % don't use monospace font for urls
\usepackage{graphicx,grffile}
\makeatletter
\def\maxwidth{\ifdim\Gin@nat@width>\linewidth\linewidth\else\Gin@nat@width\fi}
\def\maxheight{\ifdim\Gin@nat@height>\textheight\textheight\else\Gin@nat@height\fi}
\makeatother
% Scale images if necessary, so that they will not overflow the page
% margins by default, and it is still possible to overwrite the defaults
% using explicit options in \includegraphics[width, height, ...]{}
\setkeys{Gin}{width=\maxwidth,height=\maxheight,keepaspectratio}
\IfFileExists{parskip.sty}{%
\usepackage{parskip}
}{% else
\setlength{\parindent}{0pt}
\setlength{\parskip}{6pt plus 2pt minus 1pt}
}
\setlength{\emergencystretch}{3em}  % prevent overfull lines
\providecommand{\tightlist}{%
  \setlength{\itemsep}{0pt}\setlength{\parskip}{0pt}}
\setcounter{secnumdepth}{0}
% Redefines (sub)paragraphs to behave more like sections
\ifx\paragraph\undefined\else
\let\oldparagraph\paragraph
\renewcommand{\paragraph}[1]{\oldparagraph{#1}\mbox{}}
\fi
\ifx\subparagraph\undefined\else
\let\oldsubparagraph\subparagraph
\renewcommand{\subparagraph}[1]{\oldsubparagraph{#1}\mbox{}}
\fi

%%% Use protect on footnotes to avoid problems with footnotes in titles
\let\rmarkdownfootnote\footnote%
\def\footnote{\protect\rmarkdownfootnote}

%%% Change title format to be more compact
\usepackage{titling}

% Create subtitle command for use in maketitle
\newcommand{\subtitle}[1]{
  \posttitle{
    \begin{center}\large#1\end{center}
    }
}

\setlength{\droptitle}{-2em}

  \title{Factorial Models, Kronecker Products, and Interaction Terms}
    \pretitle{\vspace{\droptitle}\centering\huge}
  \posttitle{\par}
    \author{Doug Hemken}
    \preauthor{\centering\large\emph}
  \postauthor{\par}
      \predate{\centering\large\emph}
  \postdate{\par}
    \date{November 22, 2018}


\begin{document}
\maketitle

\subsection{Interaction Terms and Factorial
Models}\label{interaction-terms-and-factorial-models}

We may consider a model that includes only the mean of the response as a
zero-order model, sometimes called an intercept-only model. A model with
means of several categories, parameterized as a mean and offsets to that
mean, is a model with multiple intercepts (only), and is also a
zero-order model. Classical ANOVA models are zero-order models.

A model of a response with a continuous variable includes both an
intercept and a slope. This is a first-order model, with a zero-order
term and a first-order term. Adding categorical variables to the model
adds more intercepts, or zero-order terms. Adding continuous variables
adds more slopes, or first-order terms. Such additive models are
first-order models.

An interaction term is formed as the product of two variables. A product
of categorical variables adds intercepts to the model. The interaction
of a categorical variable and a continuous variable adds slopes to the
model. In either case, the order of the model remains the same. But the
interaction formed from the product of two continuous variables adds a
second-order term to a model, a curvature.

A factorial model is formed by adding all the products of all the zero-
and first-order terms, in all combinations. If we think of the terms in
a model as its column space, then any linear model resides in a subspace
of the factorial column space. The columns of any linear model are a
subset of the columns in a full-factorial model.

\subsubsection{An Example, Two
Variables}\label{an-example-two-variables}

Consider three models

\begin{itemize}
\tightlist
\item
  \(y = b_0 + b_1x\), also expressed in matrix form as \(y = Xb\)
\item
  \(y = c_0 + c_1z\), or \(y=Zc\)
\item
  \(y = d_0 + d_1x +d_2z + d_3xz\), or \(y=Ad\)
\end{itemize}

The column space a \(A\) may be formed as an outer product of the column
spaces of \(X\) and \(Z\). Denote
\(X = [\begin{array} & 1_x & x \end{array}]\) and
\(Z = [\begin{array} & 1_z & z \end{array}]\). Then
\[X \otimes Z = [\begin{array} & 1_x & x \end{array}] \otimes[\begin{array} & 1_z & z \end{array}]\]
This operation produces a matrix with the correct column space
\[[\begin{array} & 1_x1_z & 1_xz &x1_z & xz \end{array}]=[\begin{array} & 1 & z &x & xz \end{array}]\]
(However, our row space is a subset of the outer product of the rows,
which are also produced by any Kronecker operation. So the model matrix
for a model with interaction terms is a submatrix of the Kronecker
product(s) of the zero- and first-order terms.)


\end{document}

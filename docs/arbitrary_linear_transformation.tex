\documentclass[]{article}
\usepackage{lmodern}
\usepackage{amssymb,amsmath}
\usepackage{ifxetex,ifluatex}
\usepackage{fixltx2e} % provides \textsubscript
\ifnum 0\ifxetex 1\fi\ifluatex 1\fi=0 % if pdftex
  \usepackage[T1]{fontenc}
  \usepackage[utf8]{inputenc}
\else % if luatex or xelatex
  \ifxetex
    \usepackage{mathspec}
  \else
    \usepackage{fontspec}
  \fi
  \defaultfontfeatures{Ligatures=TeX,Scale=MatchLowercase}
\fi
% use upquote if available, for straight quotes in verbatim environments
\IfFileExists{upquote.sty}{\usepackage{upquote}}{}
% use microtype if available
\IfFileExists{microtype.sty}{%
\usepackage{microtype}
\UseMicrotypeSet[protrusion]{basicmath} % disable protrusion for tt fonts
}{}
\usepackage[margin=1in]{geometry}
\usepackage{hyperref}
\hypersetup{unicode=true,
            pdftitle={Linear Transformations of Simple Linear Regression},
            pdfauthor={Doug Hemken},
            pdfborder={0 0 0},
            breaklinks=true}
\urlstyle{same}  % don't use monospace font for urls
\usepackage{graphicx,grffile}
\makeatletter
\def\maxwidth{\ifdim\Gin@nat@width>\linewidth\linewidth\else\Gin@nat@width\fi}
\def\maxheight{\ifdim\Gin@nat@height>\textheight\textheight\else\Gin@nat@height\fi}
\makeatother
% Scale images if necessary, so that they will not overflow the page
% margins by default, and it is still possible to overwrite the defaults
% using explicit options in \includegraphics[width, height, ...]{}
\setkeys{Gin}{width=\maxwidth,height=\maxheight,keepaspectratio}
\IfFileExists{parskip.sty}{%
\usepackage{parskip}
}{% else
\setlength{\parindent}{0pt}
\setlength{\parskip}{6pt plus 2pt minus 1pt}
}
\setlength{\emergencystretch}{3em}  % prevent overfull lines
\providecommand{\tightlist}{%
  \setlength{\itemsep}{0pt}\setlength{\parskip}{0pt}}
\setcounter{secnumdepth}{0}
% Redefines (sub)paragraphs to behave more like sections
\ifx\paragraph\undefined\else
\let\oldparagraph\paragraph
\renewcommand{\paragraph}[1]{\oldparagraph{#1}\mbox{}}
\fi
\ifx\subparagraph\undefined\else
\let\oldsubparagraph\subparagraph
\renewcommand{\subparagraph}[1]{\oldsubparagraph{#1}\mbox{}}
\fi

%%% Use protect on footnotes to avoid problems with footnotes in titles
\let\rmarkdownfootnote\footnote%
\def\footnote{\protect\rmarkdownfootnote}

%%% Change title format to be more compact
\usepackage{titling}

% Create subtitle command for use in maketitle
\newcommand{\subtitle}[1]{
  \posttitle{
    \begin{center}\large#1\end{center}
    }
}

\setlength{\droptitle}{-2em}

  \title{Linear Transformations of Simple Linear Regression}
    \pretitle{\vspace{\droptitle}\centering\huge}
  \posttitle{\par}
    \author{Doug Hemken}
    \preauthor{\centering\large\emph}
  \postauthor{\par}
      \predate{\centering\large\emph}
  \postdate{\par}
    \date{2018-12-11}


\begin{document}
\maketitle

{
\setcounter{tocdepth}{2}
\tableofcontents
}
\hypertarget{general-transformations}{%
\subsection{General Transformations}\label{general-transformations}}

\begin{quote}
In a linear regression problem, \(Y=XB\), an overdetermined system to be
solved for \(B\) by least squares or maximum likelihood, let \(A\) be an
arbitrary invertible linear transformation of the columns of \(X\), so
that \(X_\delta=XA\).
\end{quote}

\begin{quote}
Then the solution, \(B_\delta\) to the transformed problem,
\(Y=X_\delta B_\delta\), is \(B_\delta=A^{-1}B\).
\end{quote}

We can see this by starting with the normal equations for \(B\) and
\(B_\delta\): \[\begin{aligned}B &=(X^TX)^{-1}X^TY \\
\\
 B_\delta &=(X_\delta^TX_\delta)^{-1}X_\delta^TY \\
 &=((XA)^TXA)^{-1}(XA)^TY \\
 &=(A^TX^TXA)^{-1}A^TX^TY \\
 &=(X^TXA)^{-1}(A^T)^{-1}A^TX^TY \\
 &=(X^TXA)^{-1}X^TY \\
 &=A^{-1}(X^TX)^{-1}X^TY \\
 &=A^{-1}B \\
\end{aligned}\]

This gives us an easy way to calculate \(B_\delta\) from \(B\), and vice
versa: \[AB_\delta=B\] The linear transformation that we used on the
columns of \(A\), inverted, gives us the transformed solution.

An example in R illustrates how an arbitrary (invertible) linear
transformation produces equal fits to the data, i.e.~the same predicted
values.

\begin{verbatim}
transf <- matrix(runif(4), ncol=2) # arbitrary linear transformation

m1 <- lm(mpg ~ wt, data=mtcars)
mpg1 <- predict(m1)

m2 <- lm(mpg ~ 0 + model.matrix(m1) %*% transf, data=mtcars)
mpg2 <- predict(m2)

plot(mpg1 ~ mpg2)
\end{verbatim}

\includegraphics{arbitrary_linear_transformation_files/figure-latex/transf-1.pdf}

\begin{verbatim}
norm(as.matrix(mpg1-mpg2), "F")
\end{verbatim}

\begin{verbatim}
## [1] 9.720162e-13
\end{verbatim}

Looking at the solutions to these two models, we see the same
transformation and it's inverse allow us to convert the coefficients
directly.

\begin{verbatim}
cbind(coef(m1), coef(m2))
\end{verbatim}

\begin{verbatim}
##                  [,1]      [,2]
## (Intercept) 37.285126  180.0250
## wt          -5.344472 -128.4037
\end{verbatim}

\begin{verbatim}
transf %*% coef(m2)
\end{verbatim}

\begin{verbatim}
##           [,1]
## [1,] 37.285126
## [2,] -5.344472
\end{verbatim}

\begin{verbatim}
solve(transf) %*% coef(m1)
\end{verbatim}

\begin{verbatim}
##           [,1]
## [1,]  180.0250
## [2,] -128.4037
\end{verbatim}

\hypertarget{recentering}{%
\subsection{Recentering}\label{recentering}}

We can use this general result to think about how coefficients change
when the data are recentered.

If our model matrix \(X\) is composed of two columns vectors,
\(\vec{1}\) and \(\vec{x}\), so that
\(X=\begin{bmatrix} \vec{1} & \vec{x} \end{bmatrix}\), then we can
recenter \(\vec{x}\) with an arbitrary constant \(\mu\), as
\(\vec{x}-\mu\), using the transformation
\[A=\begin{bmatrix} 1  & -\mu \\ 0 & 1 \end{bmatrix}\] So that,
borrowing our notation from above, \(X_\delta=XA\). For an \(A\) of this
form, we have \[A^{-1}=\begin{bmatrix} 1  & \mu \\ 0 & 1 \end{bmatrix}\]
and the solution for our recentered data is \(B_\delta=A^{-1}B\).

Continuing with the example above, we can recenter \texttt{wt} to the
sample mean, and verify that this transformation

\begin{verbatim}
wtcenter <- matrix(c(1,0,-mean(mtcars$wt),1), ncol=2)

centered <- model.matrix(m1) %*% wtcenter
colMeans(centered)  # should be 1 and 0
\end{verbatim}

\begin{verbatim}
## [1] 1.000000e+00 3.469447e-17
\end{verbatim}

Next we can calculate the transformed coefficients

\begin{verbatim}
C <- solve(wtcenter)
C %*% coef(m1)
\end{verbatim}

\begin{verbatim}
##           [,1]
## [1,] 20.090625
## [2,] -5.344472
\end{verbatim}

Verify by calculating the solution using the recentered data

\begin{verbatim}
coef(lm(mpg~0+centered, data=mtcars))
\end{verbatim}

\begin{verbatim}
## centered1 centered2 
## 20.090625 -5.344472
\end{verbatim}

\hypertarget{rescaling}{%
\subsection{Rescaling}\label{rescaling}}

We can use the general result again to think about how coefficients
change when the data are rescaled.

We can rescale \(\vec{x}\) with an arbitrary constant \(\sigma\) as
\((1/\sigma)\vec{x}\) using the transformation
\[A=\begin{bmatrix} 1  & 0 \\ 0 & \frac{1}{\sigma} \end{bmatrix}\]

For an \(A\) of this form, we have
\[A^{-1}=\begin{bmatrix} 1  & 0 \\ 0 & \sigma \end{bmatrix}\]

Again, \(X_\delta=XA\) and \(B_\delta=A^{-1}B\).

In our example, we can consider coverting \texttt{wt} from thousands of
pounds to kilograms.

\begin{verbatim}
wt2kg <- matrix(c(1,0,0,453.592), ncol=2)

scaled <- model.matrix(m1) %*% wt2kg
colMeans(scaled)  # should be 1 and 1459.3
\end{verbatim}

\begin{verbatim}
## [1]    1.000 1459.319
\end{verbatim}

Next we can calculate the transformed coefficients

\begin{verbatim}
C <- solve(wt2kg)
C %*% coef(m1)
\end{verbatim}

\begin{verbatim}
##             [,1]
## [1,] 37.28512617
## [2,] -0.01178255
\end{verbatim}

Verify by calculating the solution using the recentered data

\begin{verbatim}
coef(lm(mpg~0+scaled, data=mtcars))
\end{verbatim}

\begin{verbatim}
##     scaled1     scaled2 
## 37.28512617 -0.01178255
\end{verbatim}


\end{document}
